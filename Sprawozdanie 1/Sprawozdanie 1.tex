\documentclass[a4paper,12pt]{article}

\usepackage{geometry}
\usepackage{polski}
\usepackage{amsmath}
\usepackage{makecell}
\usepackage{ragged2e}
\usepackage{hyperref}
\usepackage{array}

\hypersetup{
	colorlinks=true,
	linkcolor=black,
	filecolor=magenta,
	urlcolor=blue,
	citecolor=black
}

\urlstyle{same}

\geometry{
 a4paper,
 total={170mm,257mm},
 left=20mm,
 top=20mm,
 }
 
 \pagenumbering{roman}

\begin{document}
\begin{justify}

\begin{center}
\begin{scriptsize}
\begin{tabular}{ |m{2.5cm}|l|l|l|l|l| }
	\hline
	\makecell{Wydział: \\ FiIS} & \multicolumn{2}{|l|}{\makecell{Imię i nazwisko: \\ 1. Piotr Moszkowicz \\ 2. Wiktor Jasiński}}  & \makecell{Rok: \\ Drugi} & \makecell{Grupa: \\ PN 14:40} & \makecell{Zespół: \\ 1} \\
	\hline
	\textbf{PRACOWNIA FIZYCZNA WFiIS AGH} & \multicolumn{4}{|l|}{\makecell{Temat: Opracowanie danych pomiarowych }}  & \makecell{Nr ćwiczenia: \\ 0} \\
	\hline
	\makecell{Data wykonania: \\ 4.03.2019} & \makecell{Data oddania: \\ 11.03.2019} & \makecell{Zwrot do popr. \\ \,} & \makecell{Data oddania \\ \,} & \makecell{Data zaliczenia \\ \,} & \makecell{OCENA \\ \,} \\
	\hline
\end{tabular}
\end{scriptsize}

\vspace{2cm}

\begin{Large}
\textbf{Ćwiczenie nr 0: Opracowanie danych pomiarowych}
\end{Large}

\end{center}

\vspace{0.5cm}
\textbf{Cel ćwiczenia:} \\
\indent Zaznajomienie się z typowymi metodami opracowania danych pomiarowych przy wykorzystaniu wyników pomiarów szybkostrzelności AK47 w grze komputerowej Counter\,Strike\,Global\,Offensive.
\end{justify}

\newpage
\tableofcontents
\newpage
\pagenumbering{arabic}

\begin{justify}

\section{Wstęp Teoretyczny}
\label{theory}

Zamiast dokonywać pomiaru wahadła postanowiliśmy zmierzyć, czy szybkostrzelność broni AK47 w grze Counter Strike Global Offensive odpowiada realnej szybkostrzelności teoretycznej tegoż karabinu na Ziemi. W tym celu potrzebne jest sprecyzowanie kilku pojęć.

\subsection{Szybkostrzelność}
Szybkostrzelność - liczba strzałów, jaką dany karabin oddanej w ciągu określonego czasu (minuty). Przedstawiona jest wzorem:
\begin{equation}
n_{t} = \frac{\Delta l}{\Delta t}
\label{szybkostrzelnosc}
\end{equation}
$\Delta l$ - ilość wystrzałów \\
$\Delta t$ - czas trwania wystrzałów  \\ \, \\
W naszym przypadku często będziemy się posługiwać angielskim skrótem RPM ("Rounds per minute").  \cite{urlszybkostrzelnosc} \\
$1 RPM = 1 \frac{strzal}{min}$

\paragraph{Szybkostrzelność teoretyczna} - szybkostrzelność przy mierzeniu nie bierzemy pod uwagę fizycznych ograniczeń takich pojemność magazynów, czas oraz konieczność przeładowania broni i tym podobne. \cite{urlszybkostrzelnosc}

\paragraph{Szybkostrzelność praktyczna} - średnia szybkostrzelność jaką możemy oddać z danego typu broni w realnych warunkach eksploatacyjnych, gdy broń obsługiwana jest przez dobrze wyszkolonego strzelca. \cite{urlszybkostrzelnosc}

\subsection{Klatka filmowa} 
Klatka filmowa - obszar na taśmie filmowej, na którym zarejestrowany jest pojedynczy nieruchomy obraz. \cite{urlklatkafilmowa}

\subsection{FPS}
FPS - ("Frame per second") - miara, która określa ilość klatek na sekundę; wykorzystywana przy określaniu parametrów nagranego filmu.

\subsection{Dodatkowe oznaczenia}
Aby jednoznacznie odczytywać wyniki, przyjęliśmy poniższe dodatkowe oznaczenia: \\ \, \\
$ k_{29} $ - Numer klatki, gdy w magazynku broni znajduje się 29 kul [-] \\
$ k_{1} $	-  Numer klatki, gdy w magazynku broni znajduje się 1 kula [-] \\
$ \Delta k_{28} = | k_{1} - k_{29} | $ - Ilość klatek potrzebnych na oddanie 28 strzałów [-] \\
$ \Delta t_{28} $ - Czas potrzebny na oddanie 28 strzałów - obliczany ze wzoru \ref{szybkostrzelnosc} [s]

\section{Aparatura pomiarowa}

W przypadku naszego doświadczenia pomiary będą dokonywane na komputerze przy pomocy dwóch programów:
\begin{itemize}
\item Open Broadcast Software Studia (64bit) - wersja 23.0.1
\item Sony Vegas PRO - wersja 16.0
\end{itemize}

\section{Wykonanie}

Na początku, aby uniknąć dużych niepewności pomiarowych, ustawiliśmy grę, w taki sposób, aby ilość wyświetlanych klatek na sekundę wynosiła zawsze 60. Jest to powiązane z naszymi nagraniami - one również są wykonywane w 60 klatkach na sekundę. To ograniczenie zostało narzucone poprzez program nagrywający. \\ \, \\
Wykonujemy kilkanaście (15) pomiarów, które uzyskujemy dzięki poniższym czynnościom:
\begin{itemize}
\item Wykonanie nagrania, która przedstawia wystrzał 30 kul (pełnego magazynka).
\item Spowalniamy film maksymalnie, tak, abyśmy mogli poruszać się po nim "klatka po klatce".
\item Znalezienie klatki, w której po raz pierwszy widzimy ilość pozostałych kul jako 29 (czyli pomijamy pierwszy wystrzał, gdyż nie jesteśmy pewni, w którym momencie dokładnie nastąpił).
\item Znalezienie klatki, w której po raz pierwszy widzimy ilość pozostałych kul jako 1 (czyli pomijamy ostatni wystrzał, gdyż nie jesteśmy pewni, w którym momencie nastąpił).
\item Zapisujemy wyniki w celu przyszłej analizy.
\end{itemize}

\section{Wyniki pomiarów}

Aby dobrze zrozumieć wyniki pomiarów  polecamy zerknąć na oznaczenia wprowadzone we wstępie teoretycznym.

\begin{table}[h!]
\begin{center}
\begin{scriptsize}
\begin{tabular}{|l|l|l|l|l|}
\hline
Lp. & $k_{29}$ [-] & $k_{1}$ [-] & $\Delta k$ [-] & $\Delta t$ [s] \\
\hline
1 & 89 & 257 & 168 & 2,80s \\
2 & 91 & 259 & 168 & 2,80s \\
3 & 76 & 244 & 168 & 2,80s  \\
4 & 61 & 228 & 167 & 2,78s \\
5 & 64 & 232 & 168 & 2,80s \\
6 & 74 & 242 & 168 & 2,80s \\
7 & 108 & 276 & 168 & 2,80s \\
8 & 109 & 277 & 168 & 2,80s \\
9 & 47 & 214 & 167 & 2,78s \\
10 & 64 & 232 & 168 & 2,80s \\
11 & 51 & 219 & 168 & 2,80s  \\
12 & 81 & 249 & 168 & 2,80s \\
13 & 64 & 234 & 170 & 2,83s \\
14 & 72 & 239 & 167 & 2,78s \\
15 & 73 & 242 & 169 & 2,82s \\
\hline
\end{tabular}
\caption{Tabela zawierająca wyniki pomiarów istotnych klatek}
\label{table:1}
\end{scriptsize}
\end{center}
\end{table}

\section{Opracowanie wyników}

\subsection{Błędy grube}
Nasze pomiary nie zawierają błędów grubych - specyfika doświadczenia nie pozwala na uzyskanie takiego błędu.

\subsection{Niepewność pomiaru (typu A)}
Zgodnie z informacjami zawartymi w pomocy przy opracowaniu danych pomiarowych \cite{urlniepewnosci} niepewność pomiaru czasu typu A obliczamy jako estymator odchylenia standardowego średniej zgodnie ze wzorem: \\
\begin{equation}
u(\Delta t) = \sqrt{\frac{\sum(t_{i} - \overline{t})^2}{n (n-1)}}
\end{equation}

W naszym przypadku $u(\Delta t) = 0.045s $.

\subsection{Niepewność pomiaru (typu B)}
W związku z tym, iż gra czasami (mimo specjalnych ustawień) wyświetlała obraz w 59 klatkach na sekundę, przyjęliśmy, iż niepewność pomiarowa (ilości klatek) wynosi  3. Niepewność obliczenia ilości klatek jest zerowa, stąd: 
$u_{b}(\Delta l) = 0$ oraz $u_{b}(\Delta t) = \frac{0.05}{\sqrt{3}} = 0.02s$ \\
Na mocy prawa przenoszenia niepewności pomiarowych możemy finalnie wyliczyć niepewność typu B: 

$u_{b}(\Delta t) = \sqrt{(\frac{1}{\Delta t} * u(\Delta l)^2 + (\frac{\Delta l}{(\Delta t)^2} * u(\Delta t))^2} = 0.08s$

\subsection{Na podstawie uzyskanych pomiarów oblicz szybkostrzelność}

\begin{table}[h!]
\begin{center}
\begin{tabular}{|m{3cm}|m{8cm}|}
\hline
Nr pomiaru & Szybkostrzelność teoretyczna (wzór \ref{szybkostrzelnosc}) [RPM] \\
\hline
1 & 600 \\
2 & 600 \\
3 & 600 \\
4 & 603 \\
5 & 600 \\
6 & 600 \\
7 & 600 \\
8 & 600 \\
9 & 603 \\
10 & 600 \\
11 & 600 \\
12 & 600 \\
13 & 593 \\
14 & 603 \\
15 & 596 \\
\hline
\end{tabular}
\caption{Tabela zawierająca obliczoną szybkostrzelność na podstawie danych z każdego pomiaru}
\label{table:2}
\end{center}
\end{table}

\subsection{Czy uzyskana wartość szybkostrzelności jest zgodna, w granicach niepewności z wartością rzeczywistą?}

Odpowiedź na to pytanie znajduje się w sekcji wyniki.

\section{Wyniki}

Naszym wynikiem jest uśredniona szybkostrzelność otrzymana na podstawie wyników z tabeli \ref{table:2} zgodnie ze wzorem:
$\overline{s} = \frac{1}{15} * \sum_{n=1}^{15} s_{i}$  \\ \, \\

Uzyskany wynik 600.01 RPM.

\section{Wnioski}

Obliczmy niepewność rozszerzoną daną wzorem: $U(n_{t}) = k * u_{b}(n_{t}) = 2 * \frac{60 * 0.08}{28} = 0.17 RPM  $
Zgodnie ze źródłami przedstawionymi w bibliografii szybkostrzelność teoretyczna karabinka AK47 powinna wynosić 600 RPM \cite{urlkalach}. Biorąc pod uwagę tę wartość możemy wykonać test statystyczny:
\begin{center}
$| n - n_{t} | \leq U(n_{t}) $ \\
$| 600 - 600.01 | \leq 0.17 $ \\
$0.01 \leq 0.17 $ \\
\end{center}

Zgodnie z testem statystycznym otrzymaliśmy poprawny wynik w granicy błędu pomiarowego.

\section{Bibliografia}

\begingroup
\renewcommand{\section}[2]{}%
\begin{thebibliography}{}
\bibitem{urlszybkostrzelnosc} \url{https://pl.wikipedia.org/wiki/Szybkostrzelność}
\bibitem{urlklatkafilmowa} \url{https://encyklopedia.pwn.pl/haslo/klatka-filmowa;3922730.html}
\bibitem{urlkalach} \url{https://pl.wikipedia.org/wiki/Karabinek_AK}
\bibitem{urlniepewnosci} \url{http://www.fis.agh.edu.pl/~pracownia_fizyczna/pomoce/OpracowanieDanychPomiarowych.pdf}
\end{thebibliography}
\endgroup

\end{justify}

\end{document}